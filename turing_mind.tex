\documentclass[10pt]{article} % Starting with 10pt as a base

% --- Preamble: Packages and Global Settings ---
\usepackage[utf8]{inputenc} % Input encoding (important for direct copy-paste of text)
\usepackage[T1]{fontenc}    % Font encoding for proper glyphs and hyphenation
\usepackage{tgschola}       % For TeX Gyre Schola font
\usepackage{amsmath}        % For math environments (e.g., align)
\usepackage{amssymb}        % For additional math symbols
\usepackage{graphicx}       % For including images (if any)
\usepackage{setspace}       % For line spacing control
\usepackage{titlesec}       % For fine-grained control over section titles
\usepackage{fancyhdr}       % For custom headers and footers
\usepackage{soul}           % For letter spacing (\so)
\usepackage{enumitem}       % For customizing list environments (e.g., (i), (ii))

% --- Page Geometry (from your precise measurements) ---
\usepackage[
    paperwidth=141mm,   % Width of the physical paper
    paperheight=231mm,  % Height of the physical paper
    left=22mm,          % Distance from left paper edge to text block
    right=22mm,         % Distance from right paper edge to text block
    top=22mm,           % Distance from top paper edge to text block
    bottom=34mm,        % Distance from bottom paper edge to text block
    % With these, textwidth=97mm and textheight=175mm are automatically calculated by geometry.
]{geometry}

% --- Header and Footer Configuration ---
% Define default page style using fancyhdr
\pagestyle{fancy}
\fancyhf{} % Clear all header and footer fields

% First page style (unique header/footer)
\fancypagestyle{firstpage}{
    \fancyhf{} % Clear existing settings for this page
    % Confirmed from scan: VOL. LIX. is on one line, No. 236.] is on the next, slightly indented.
    % The font size is indeed very small.
    \fancyhead[L]{\normalfont\fontsize{6.5pt}{7.5pt}\selectfont VOL. LIX.\\ \hspace*{0.5em}No. 236.]} % Top left (Vol/No info), with indent on second line
    \fancyhead[R]{\normalfont\fontsize{6.5pt}{7.5pt}\selectfont [October, 1950} % Top right (Date)
    \fancyfoot[L]{\normalfont\small 28} % Bottom left (Journal series number)
    \fancyfoot[R]{\normalfont\small 433} % Bottom right (Page number)
    \renewcommand{\headrulewidth}{0pt} % No rule under header
    \renewcommand{\footrulewidth}{0pt} % No rule above footer
}

% General page style for subsequent pages (alternating running heads and page numbers)
\fancypagestyle{mainpage}{
    \fancyhf{} % Clear existing settings
    % Left Even pages (e.g., page 434): Page number on left, Author on right
    \fancyhead[LE]{\normalfont\small \thepage \ A. M. TURING :} % Left Even
    % Right Odd pages (e.g., page 435): Article Title on left, Page number on right
    \fancyhead[RO]{\normalfont\small COMPUTING MACHINERY AND INTELLIGENCE \thepage} % Right Odd
    \renewcommand{\headrulewidth}{0pt} % No rule under header
    \renewcommand{\footrulewidth}{0pt} % No rule above footer
}
\pagestyle{mainpage} % Set this as the default style for all pages (first page overrides it)

% --- Customizing Article Title and Section Titles ---
% Custom command for the main article title (I. COMPUTING MACHINERY AND INTELLIGENCE)
% Now centered and slightly larger
\newcommand{\articletitle}[1]{%
    \vspace*{0.7cm} % Adjust vertical spacing above title
    \begin{center}% Center the title
        {\normalfont\LARGE\textbf{#1}}% Slightly larger, bold, uppercase
    \end{center}%
    \par\vspace{0.2cm}%
}

% Custom command for the author line (BY A. M. TURING)
% Now centered, and explicitly in small caps
\newcommand{\articleauthor}[1]{%
    \begin{center}% Center the author
        {\normalfont#1}% Content passed will be small caps
    \end{center}%
    \par\vspace{0.5cm}%
}

% Global line spacing for body text
\setstretch{1.15} % This value might need fine-tuning for exact line count per page.

% --- Document Start ---
\begin{document}
\thispagestyle{firstpage} % Apply the unique first page style to the first page only

% --- Top Journal Information (handled by fancypagestyle{firstpage}) ---

% --- Main Journal Title Block ---
\vspace*{1.2cm} % Adjust this to push "MIND" down to match the original
\begin{center}
    % Applying letter spacing with \so from 'soul' package
    % Keeping it bold and huge, as that matches the original's heavy stroke.
    {\Huge\textbf{\so{MIND}}}
\end{center}

\vspace*{0.5cm} % Space between MIND and A QUARTERLY REVIEW
\begin{center}
    {\large A QUARTERLY REVIEW} \\
    \vspace{0.1cm} % Small space between lines
    {\normalsize OF} \\
    \vspace{0.1cm} % Small space between lines
    {\large PSYCHOLOGY AND PHILOSOPHY}
\end{center}

% Decorative separator
\vspace*{0.5cm} % Space below philosophy text
\begin{center}
    \rule{3cm}{0.75pt} % Horizontal rule mimicking the decorative element
\end{center}
\vspace*{0.7cm} % Space below separator

% --- Article Title and Author ---
\articletitle{I.\textemdash COMPUTING MACHINERY AND INTELLIGENCE} % Use \textemdash for the long dash (em-dash)
\articleauthor{\textsc{BY A. M. TURING}} % Now BY A. M. TURING is passed as small caps

% --- Article Content Starts ---
% Section 1
% Section header: number normal, rest italic.
\noindent\normalfont 1. \textit{The Imitation Game.}
\vspace{0.5\baselineskip} % Small space below the section title

I \textsc{PROPOSE} to consider the question, `Can machines think?' This should begin with definitions of the meaning of the terms `machine' and `think'. The definitions might be framed so as to reflect so far as possible the normal use of the words, but this attitude is dangerous. If the meaning of the words `machine' and `think' are to be found by examining how they are commonly used it is difficult to escape the conclusion that the meaning and the answer to the question, `Can machines think?' is to be sought in a statistical survey such as a Gallup poll. But this is absurd. Instead of attempting such a definition I shall replace the question by another, which is closely related to it and is expressed in relatively unambiguous words.

The new form of the problem can be described in terms of a game which we call the `imitation game'. It is played with three people, a man (A), a woman (B), and an interrogator (C) who may be of either sex. The interrogator stays in a room apart from the other two. The object of the game for the interrogator is to determine which of the other two is the man and which is the woman. He knows them by labels X and Y, and at the end of the game he says either `X is A and Y is B' or `X is B and Y is A'. The interrogator is allowed to put questions to A and B thus:

\vspace{0.5\baselineskip} % Space before dialogue starts
\textbf{C}: Will X please tell me the length of his or her hair ?\par
\vspace{0.2\baselineskip} % Small space between dialogue turns
Now suppose X is actually A, then A must answer. It is A's
% --- End of Original PDF page 433 (my page 1) ---

% --- Content for Original PDF page 434 (my page 2) ---
object in the game to try and cause C to make the wrong identification. His answer might therefore be
\vspace{0.2\baselineskip} % Small space before quoted answer
``My hair is shingled, and the longest strands are about nine inches long.''% Corrected curly quotes (no \par after this if it's part of the flowing text)
\vspace{0.5\baselineskip} % Space after quoted answer

In order that tones of voice may not help the interrogator the answers should be written, or better still, typewritten. The ideal arrangement is to have a teleprinter communicating between the two rooms. Alternatively the question and answers can be repeated by an intermediary. The object of the game for the third player (B) is to help the interrogator. The best strategy for her is probably to give truthful answers. She can add such things as ``I am the woman, don't listen to him!'' to her answers, but it will avail nothing as the man can make similar remarks.

We now ask the question, `What will happen when a machine takes the part of A in this game?'` Will the interrogator decide wrongly as often when the game is played like this as he does when the game is played between a man and a woman? These questions replace our original, `Can machines think?'`

% Section 2
% Section header: number normal, rest italic.
\vspace{0.5\baselineskip} % Space before new section
\noindent\normalfont 2. \textit{Critique of the New Problem.}
\vspace{0.5\baselineskip} % Small space below the section title

As well as asking, `What is the answer to this new form of the question,` one may ask, `Is this new question a worthy one to investigate?'` This latter question we investigate without further ado, thereby cutting short an infinite regress.

The new problem has the advantage of drawing a fairly sharp line between the physical and the intellectual capacities of a man. No engineer or chemist claims to be able to produce a material which is indistinguishable from the human skin. It is possible that at some time this might be done, but even supposing this invention available we should feel there was little point in trying to make a `thinking machine'` more human by dressing it up in such artificial flesh. The form in which we have set the problem reflects this fact in the condition which prevents the interrogator from seeing or touching the other competitors, or hearing their voices. Some other advantages of the proposed criterion may be shown up by specimen questions and answers. Thus:

\vspace{0.5\baselineskip} % Space before dialogue
\textbf{Q}: Please write me a sonnet on the subject of the Forth Bridge.\par
\textbf{A}: Count me out on this one. I never could write poetry.\par
\vspace{0.2\baselineskip} % Small space between dialogue turns
\textbf{Q}: Add 34957 to 70764\par
\textbf{A}: (Pause about 30 seconds and then give as answer) 105621.\par
\vspace{0.2\baselineskip} % Small space between dialogue turns
\textbf{Q}: Do you play chess ?\par
\textbf{A}: Yes.
% --- End of Original PDF page 434 (my page 2) ---

% --- Content for Original PDF page 435 (my page 3) ---
% The original page 435 starts with the rest of the chess dialogue.
\vspace{0.5\baselineskip} % Add a bit of space before the continuation of Q&A
\textbf{Q}: I have K at my K1, and no other pieces. You have only K at K6 and R at R1. It is your move. What do you play ?\par
\textbf{A}: (After a pause of 15 seconds) R-R8 mate.\par

% Section 3 now correctly placed after the full Q&A from page 435
\vspace{0.5\baselineskip} % Space before new section
\noindent\normalfont 3. \textit{The Machines concerned in the Game.}
\vspace{0.5\baselineskip} % Small space below the section title

The question which we put in \S 1 will not be quite definite until we have specified what we mean by the word `machine'. It is natural that we should wish to permit every kind of engineering technique to be used in our machines. We also wish to allow the possibility than an engineer or team of engineers may construct a machine which works, but whose manner of operation cannot be satisfactorily described by its constructors because they have applied a method which is largely experimental. Finally, we wish to exclude from the machines men born in the usual manner. It is difficult to frame the definitions so as to satisfy these three conditions. One might for instance insist that the team of engineers should be all of one sex, but this would not really be satisfactory, for it is probably possible to rear a complete individual from a single cell of the skin (say) of a man. To do so would be a feat of biological technique deserving of the very highest praise, but we would not be inclined to regard it`as a case of `constructing a thinking machine'. This prompts us to abandon the requirement that every kind of technique should be permitted. We are the more ready to do so in view of the fact that the present interest in `thinking machines' has been aroused by a particular kind of machine, usually called an `electronic computer' or `digital computer'. Following this suggestion we only permit digital computers to take part in our game.

This restriction appears at first sight to be a very drastic one. I shall attempt to show that it is not so in reality. To do this necessitates a short account of the nature and properties of these computers.

It may also be said that this identification of machines with digital computers, like our criterion for `thinking', will only be unsatisfactory if (contrary to my belief), it turns out that digital computers are unable to give a good showing in the game.

There are already a number of digital computers in working order, and it may be asked, `Why not try the experiment straight away? It would be easy to satisfy the conditions of the game. A number of interrogators could be used, and statistics compiled to show how often the right identification was given.' The short answer is that we are not asking whether all digital computers would do well in the game nor whether the computers at present available would do well, but whether there are imaginable computers which would do well. But this is only the short answer. We shall see this question in a different light later.
% --- End of Original PDF page 436 (my page 4) ---

% --- Content for Original PDF page 437 (my page 5) ---
% Section 4
\vspace{0.5\baselineskip} % Space before new section
\noindent\normalfont 4. \textit{Digital Computers.}
\vspace{0.5\baselineskip} % Small space below the section title

The idea behind digital computers may be explained by saying that these machines are intended to carry out any operations which could be done by a human computer. The human computer is supposed to be following fixed rules; he has no authority to deviate from them in any detail. We may suppose that these rules are supplied in a book, which is altered whenever he is put on to a new job. He has also an unlimited supply of paper on which he does his calculations. He may also do his multiplications and additions on a `desk machine', but this is not important.
If we use the above explanation as a definition we shall be in danger of circularity of argument. We avoid this by giving an outline of the means by which the desired effect is achieved. A digital computer can usually be regarded as consisting of three parts:
\begin{enumerate}[label=(\roman*)]
    \item Store.
    \item Executive unit.
    \item Control.
\end{enumerate}
The store is a store of information, and corresponds to the human computer's paper, whether this is the paper on which he does his calculations or that on which his book of rules is printed. In so far as the human computer does calculations in his head a part of the store will correspond to his memory.
The executive unit is the part which carries out the various individual operations involved in a calculation. What these individual operations are will vary from machine to machine. Usually fairly lengthy operations can be done such as `Multiply 3540675445 by 7076345687'` but in some machines only very simple ones such as `Write down 0'` are possible.
We have mentioned that the `book of rules'` supplied to the computer is replaced in the machine by a part of the store. It is then called the `table of instructions'`. It is the duty of the control to see that these instructions are obeyed correctly and in the right order. The control is so constructed that this necessarily happens.
The information in the store is usually broken up into packets of moderately small size. In one machine, for instance, a packet might consist of ten decimal digits. Numbers are assigned to the parts of the store in which the various packets of information are stored, in some systematic manner. A typical instruction might say-
\vspace{0.2\baselineskip} % Small space before quoted instruction
``Add the number stored in position 6809 to that in 4302 and put the result back into the latter storage position.''% Corrected curly quotes (no \par after this if it's part of the flowing text)
\vspace{0.5\baselineskip} % Space after quoted instruction

Needless to say it would not occur in the machine expressed in English. It would more likely be coded in a form such as 6809430217. Here 17 says which of various possible operations is to be performed on the two numbers. In this case the operation is that described above, viz. ``Add the number....''` It will be noticed that the instruction takes up 10 digits and so forms one packet of information, very conveniently. The control will normally take the instructions to be obeyed in the order of the positions in which they are stored, but occasionally an instruction such as
% --- End of Original PDF page 437 (my page 5) ---

% --- Content for Original PDF page 438 (my page 6) ---
\vspace{0.5\baselineskip} % Space before quoted instruction
``Now obey the instruction stored in position 5606, and continue from there''% Removed extra quote char at end
\vspace{0.5\baselineskip} % Space after quoted instruction
may be encountered, or again
\vspace{0.5\baselineskip} % Space before quoted instruction
``If position 4505 contains 0 obey next the instruction stored in 6707, otherwise continue straight on.''% Removed extra quote char at end
\vspace{0.5\baselineskip} % Space after quoted instruction
Instructions of these latter types are very important because they make it possible for a sequence of operations to be repeated over and over again until some condition is fulfilled, but in doing so to obey, not fresh instructions on each repetition, but the same ones over and over again. To take a domestic analogy. Suppose Mother wants Tommy to call at the cobbler's every morning on his way to school to see if her shoes are done, she can ask him afresh every morning. Alternatively she can stick up a notice once and for all in the hall which he will see when he leaves for school and which tells him to call for the shoes, and also to destroy the notice when he comes back if he has the shoes with him.

The reader must accept it as a fact that digital computers can be constructed, and indeed have been constructed, according to the principles we have described, and that they can in fact mimic the actions of a human computer very closely.

The book of rules which we have described our human computer as using is of course a convenient fiction. Actual human computers really remember what they have got to do. If one wants to make a machine mimic the behaviour of the human computer in some complex operation one has to ask him how it is done, and then translate the answer into the form of an instruction table. Constructing instruction tables is usually described as `programming'. To `programme a machine to carry out the operation A'` means to put the appropriate instruction table into the machine so that it will do A.

An interesting variant on the idea of a digital computer is a `digital computer with a random element'`. These have instructions involving the throwing of a die or some equivalent electronic process; one such instruction might for instance be, ``Throw the die and put the resulting number into store 1000''.` Sometimes such a machine is described as having free will (though I would not use this phrase myself). It is not normally possible to determine from observing a machine whether it has a random element, for a similar effect can be produced by such devices as making the choices depend on the digits of the decimal for $\pi$.

Most actual digital computers have only a finite store. There is no theoretical difficulty in the idea of a computer with an unlimited store. Of course only a finite part can have been used at any one time. Likewise only a finite amount can have been
% --- End of Original PDF page 438 (my page 6) ---

% --- Content for Original PDF page 439 (my page 7) ---
constructed, but we can imagine more and more being added as required. Such computers have special theoretical interest and will be called infinitive capacity computers.

The idea of a digital computer is an old one. Charles Babbage, Lucasian Professor of Mathematics at Cambridge from 1828 to 1839, planned such a machine, called the Analytical Engine, but it was never completed. Although Babbage had all the essential ideas, his machine was not at that time such a very attractive prospect. The speed which would have been available would be definitely faster than a human computer but something like 100 times slower than the Manchester machine, itself one of the slower of the modern machines. The storage was to be purely mechanical, using wheels and cards.

The fact that Babbage's Analytical Engine was to be entirely mechanical will help us to rid ourselves of a superstition. Importance is often attached to the fact that modern digital computers are electrical, and that the nervous system also is electrical. Since Babbage's machine was not electrical, and since all digital computers are in a sense equivalent, we see that this use of electricity cannot be of theoretical importance. Of course electricity usually comes in where fast signalling is concerned, so that it is not surprising that we find it in both these connections. In the nervous system chemical phenomena are at least as important as electrical. In certain computers the storage system is mainly acoustic. The feature of using electricity is thus seen to be only a very superficial similarity. If we wish to find such similarities we should look rather for mathematical analogies of function.

% Section 5
\vspace{0.5\baselineskip} % Space before new section
\noindent\normalfont 5. \textit{Universality of Digital Computers.}
\vspace{0.5\baselineskip} % Small space below the section title

The digital computers considered in the last section may be classified amongst the `discrete state machines'. These are the machines which move by sudden jumps or clicks from one quite definite state to another. These states are sufficiently different for the possibility of confusion between them to be ignored. Strictly speaking there are no such machines. Everything really moves continuously. But there are many kinds of machine which can profitably be thought of as being discrete state machines. For instance in considering the switches for a lighting system it is a convenient fiction that each switch must be definitely on or definitely off. There must be intermediate positions, but for most purposes we can forget about them. As an example of a discrete state machine we might consider a wheel which clicks round through 120$^\circ$ once a second, but may be stopped by a lever which can be operated from outside; in addition a lamp is to light in one of the positions of the wheel. This machine could be described abstractly as follows. The internal state of the machine (which is described by the position of the wheel) may be q$_1$, q$_2$ or q$_3$. There is an input signal i$_0$, or i$_1$ (position of lever). The internal state at any moment is determined by the last state and input signal according to the table
\vspace{0.5\baselineskip} % Space before table

\begin{center}
    \normalfont % Ensure normal font within table
    \textbf{Last State} \\
    \begin{tabular}{|c|c|c|c|}
        \hline
        \multicolumn{1}{|c|}{} & \textbf{q$_1$} & \textbf{q$_2$} & \textbf{q$_3$} \\
        \hline
        \textbf{i$_0$} & q$_2$ & q$_3$ & q$_1$ \\
        \cline{2-4}
        \textbf{Input} & & & \\ % This row is for "Input" label, so it's empty otherwise
        \cline{1-1} % Only draw line under "Input"
        \textbf{i$_1$} & q$_1$ & q$_2$ & q$_3$ \\
        \hline
    \end{tabular}
\end{center}
\vspace{0.5\baselineskip} % Space after table

The output signals, the only externally visible indication of the internal state (the light) are described by the table
\vspace{0.5\baselineskip} % Space before table

\begin{center}
    \normalfont % Ensure normal font within table
    \begin{tabular}{ccc}
        \textbf{State} & \textbf{q$_1$} & \textbf{q$_2$} \textbf{q$_3$} \\
        \textbf{Output} & \textbf{o$_0$} & \textbf{o$_0$} \textbf{o$_1$} \\ % CORRECTED: This line
    \end{tabular}
\end{center}
\vspace{0.5\baselineskip} % Space after table

This example is typical of discrete state machines. They can be described by such tables provided they have only a finite number of possible states.

It will seem that given the initial state of the machine and the input signals it is always possible to predict all future states. This is reminiscent of Laplace's view that from the complete state of the universe at one moment of time, as described by the positions and velocities of all particles, it should be possible to predict all future states. The prediction which we are considering is, however, rather nearer to practicability than that considered by Laplace. The system of the `universe as a whole'` is such that quite small errors in the initial conditions can have an overwhelming effect at a later time. The displacement of a single electron by a billionth of a centimetre at one moment might make the difference between a man being killed by an avalanche a year later, or escaping. It is an essential property of the mechanical systems which we have called `discrete state machines'` that this phenomenon does not occur. Even when we consider the actual physical machines instead of the idealised machines, reasonably accurate knowledge of the state at one moment yields reasonably accurate knowledge any number of steps later.

\end{document}
